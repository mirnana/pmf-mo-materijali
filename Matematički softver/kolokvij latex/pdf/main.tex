\documentclass[10pt]{scrartcl}
\usepackage[utf8]{inputenc}
\usepackage{amsmath,amssymb,amsthm}
\usepackage[croatian]{babel}
\usepackage{csquotes}
\MakeOuterQuote{"}
\usepackage[unicode]{hyperref}


\usepackage{thmtools}
\declaretheorem{teorem}
\declaretheorem[sibling=teorem]{hipoteza}
\declaretheorem[style=definition,sibling=teorem, name=Pokušaj rješenja]{pokusaj}

\usepackage{xstring}
\usepackage{mathtools}

\begin{document}
\pagenumbering{gobble} 


\renewcommand{\today}{\the\day. \MakeUppercase{\romannumeral\month}. '\StrGobbleLeft{\the\year}{2}.} % izvor: kolegica Kristina Mikulec mi je objasnila kako ovo napraviti *prije kolokvija*
\title{Matematički softver}
\subtitle{drugi kolokvij}
\date{Praktikum 2, \today}
\author{Mirna Imrović}

\maketitle

\begin{tabular}[width=\textwidth]{r @{:\ } p{0.8\textwidth}}
    \underline{Prvi zadatak} & 
    P\'al Erd\H{o}s (1913.--1996.) rođen je u Austro-Ugarskoj, a umro u Warszawi, u Poljskoj. Volio je problem Stanis\l{}awa Ulama o funkciji
    $c(x) := 
    \begin{cases}  
    \dfrac{x}{2}, & x\text{ paran}\\
    3\cdot x+1, & \text{ inače}
    \end{cases}.$ 
    \bigskip\\
    \underline{Drugi zadatak} & 
    \begin{hipoteza}
    Za sve $t\in\mathbb N$ postoji $\ell\in\mathbb N$ tako da je $c^\ell(t)= \underbrace{c\big(c(\cdots c}_{\ell\text{ poziva}}(t)\cdots )\big)=1$.
    \end{hipoteza}
    
    \begin{pokusaj}
     Možda je $\limsup\limits_n\dfrac{\min k}{n}=+\infty$.
    \end{pokusaj}
    
    Ovo se čini malo lakšim za dokazati od hipoteze 1.
    \bigskip\\
    
    \underline{Treći zadatak} &
    \[
    \mathcal B = 
    \left\{B_\alpha\in\mathcal T \mid (\forall\;U \in \mathcal T)\ \exists (B_\beta)_\beta 
    \left(U=\bigcup\nolimits_\beta B_\beta\right)
    \right\}
    \]
    
    \begin{equation}
        \begin{aligned}
            A\subset B\setminus C
            &\not\Rightarrow A\cup C \subseteq B\\
            & \xRightarrow[]{\text{za vježbu}} A\cap C = \emptyset % https://tex.stackexchange.com/questions/85113/xrightarrow-text
        \end{aligned} \tag{$\ast$}
    \end{equation}
\end{tabular}


\end{document}